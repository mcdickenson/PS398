\documentclass[12pt,letter]{article}
\usepackage{amsmath}
\usepackage{setspace}
\onehalfspacing
\usepackage{amssymb}
\usepackage{dcolumn}
\usepackage{fullpage}
\usepackage{graphicx}
\usepackage{verbatim}
\usepackage[longnamesfirst]{natbib}
\usepackage{rotating}
\usepackage{multirow}
\usepackage{tabularx}
\usepackage{array}
\usepackage{enumerate}
\usepackage{hyperref}

\hypersetup{
    bookmarks=true,         % show bookmarks bar?
    unicode=false,          % non-Latin characters in Acrobat�s bookmarks
    pdftoolbar=true,        % show Acrobat�s toolbar?
    pdfmenubar=true,        % show Acrobat�s menu?
    pdffitwindow=false,     % window fit to page when opened
    pdfstartview={FitH},    % fits the width of the page to the window
    % pdftitle={Leadership Transitions and Violence in Mexican Drug Trafficking Organizations, 2006-2010},    % title
    pdfauthor={Matt Dickenson},     % author
    % pdfsubject={Violence},   % subject of the document
    pdfcreator={Matt Dickenson},   % creator of the document
    pdfproducer={Producer}, % producer of the document
    pdfkeywords={keywords}, % list of keywords
    pdfnewwindow=true,      % links in new window
    colorlinks=true,       % false: boxed links; true: colored links
    linkcolor=black,          % color of internal links
    linkbordercolor={0 0 0},  %border color
    citebordercolor={0 0 0},
    citecolor=black,        % color of links to bibliography
    filecolor=black,      % color of file links
    urlcolor=blue,         % color of external links
}


\begin{document}

\title{Computational Frameworks for Social Science}

\author{Matt Dickenson\thanks{Notes for Josh Cutler's PS 398 course, Duke
    University, Spring 2012.}}

\date{\text{Current as of: }\today}

\maketitle

\section{Introduction}

\begin{table}[h!]
\begin{center}
\caption{The Software World According to Josh Cutler}
% people will tell you it doesn't look like this but it does
\begin{tabular}{ccc}
\\
Software Engineers & Computer Scientists & Hackers/Coders \\
\hline
IBM & Google or PhD & Start-ups \\
Wear Ties & Beards & Build Apps \\
\end{tabular}
\end{center}
\end{table}
% All three of these are useful, and we will draw from each of
% them. Today we'll talk about source control, which is firmly in the
% software engineering world. 

\subsection{Source Control} We will use \texttt{Git} as our source
control. \texttt{Git} was a sea change in CS. Git can save versions
$A$ and $A+1$ simultaneously, so that you can work with multiple
versoins of code and cooperate with other people. To use:
\\
\texttt{git init} \\ % this creates a ``change set'', which is a list of
                     % files and their changes
\texttt{git add text.txt} \\ % tells git to track that file
\texttt{commit -m ``first commit''} \\ % committing a change set indicates a unitary block of
            % code. more on this later

Once we commit, this flushes the change set and creates ``first
commit'' in the git log.

~\\
\texttt{git add text.text} \\ % adds to the change set
\texttt{commit -m ``after edits''} \\ % adds ``after edits'' to the git log

To fully preserve for posterity: % what if my laptop blows up?
~\\ \texttt{git push .} \\ % sends something to git

To create a new version of yourself: % matt 2.0
~\\ \texttt{git pull .} \\ % recreate from git

The key idea here is that you can push to/pull from arbitrary
places. There is no hierarchy for what is the True version. 

\subsubsection{A Word on Branching}
Branching is how code gets built. Branching is a way to expand a
codebase while keeping an earlier version stable. When someone talks about the ``hot
new $\alpha$ version'', they are talking about a branch. For instance,
we may push a package to CRAN (v1.0), then start work on an update
(v2.0). In the meantime, we find out a bug in v1.0 that we need to
fix. With branching, this isn't a problem. 

% draw lots of pretty pictures or project GitHub onto a screen

\subsubsection{Naming Versions}
Consider a version $X.Y.Z$. We commit $X.Y.Z+1$ when we have fixed a
bug. We commit $X.Y+1.Z$ when we have added a feature. We commit $X+1.Y.Z$
when we want to charge people more money. 

\subsubsection{Commit History}
A basic rule of thumb is: if you can't describe what you did in a
single line, it's too big a jump. 

Sometimes in professional software you will see ``commit wars'' where
a couple of people just push the same junk back and forth until
someone fires them. For example, one guy may indent his Python code
with two spaces, while another uses four. Agree on this
early. Communicate. 


\subsection{Testing}

Automated testing, that is. We're going to use code that we write to
test our other code. Specifically, we're going to use ``unit tests''
(as opposed to ``integration testing''). Testing is
contentious--people will fight about, leave companies over, or start
companies because of this. 

A motivating example is BDD vs. TDD--behavioral driven design
\emph{versus} test driven design. BDD people are masturbatory and
annoying. TDD people write tests first and then code. They write a
series of tests that suitable code should pass, and then design
code. Others write code and then test it, but they think of the same
test cases they did when they were coding, so they may overlook
problems. Your choice will likely depend on how long you are going to
have to maintain the code. Technically, TDD code has to fail a test
before there is a bug.

But we are talking at a non-philosophical level about unit
tests. Reasons to use TDD:

\begin{itemize}
\item if someone else inherits your code, it helps them to have a test
  suite to detect hairy portions of code
\item it helps readers to see what the code is actually supposed to do
\item you can refactor (rewrite) your code with more confidence, since
  you can quickly check all known relevant tests (without TDD, you
  will undoubtedly introduce bugs; with TDD you will be at least as
  good as you were before) % to the extent that you want to make code
                           % that is useful to others, you will have
                           % to rewrite
\item it helps you be lazy--most people don't have enough mental RAM
  to remember everything a program does
\item you will write better code if you know you have to test it
\item if you truly understand what code is supposed to do, you should
  be able to assert what output it should give you
\end{itemize}

Python is dynamic (duck) code. The important thing to remember is that
there are more bugs that can be introduced than, for example, C++. To
see what we're talking about, let's actually write some code.

% inclass.py and inclasstest.py

The two things that we check for are \emph{correctness} and
\emph{robustness}. Correctness asks, ``does the code do what we want it
to when we give it the right input?'' Robustness asks, ``how does the
code handle unexpected inputs?'' In Python's \texttt{unittest} library, a test can have three
outcomes: ``.'' (pass), ``F'' (fail), or ``E'' (error). The key point:
have a failure plan. This is important, for example, when writing a
web scraper. HTML is supposed to look a certain way, but in reality it
never does. 

The process is iterative: 
\begin{enumerate}
\item see a bug
\item write a test
\item fix the code so that it satisfies the test
\item repeat
\end{enumerate}

Again, the argument for using code is that your mind will be more open
(and thus write more appropriate tests) before you develop the tunnel
vision that comes with writing code in a certain way. 

\newpage
\section{Object-Oriented Programming}
% begin with code review from last week
% tests that would fail (but no one used): 
% ``Dr.''
% ``Some ellipses...''
% Note that tests must start with ``test_''
% This is so that you can write helper functions that don't start with
% test
% Coder talk: \texttt{foo} and \texttt{bar} are vernacular

\subsection{Mini-Homework}
Restructure homeworks on github so that the root looks like \texttt{/HW1} and
use \texttt{/HW2} for the next one.

\subsection{What is OOP?}
\begin{table}[h!]
\begin{center}
\caption{Three Styles of Programming}
\begin{tabular}{ccc}
\\
Procedural & Functional & Object-Oriented (OOP) \\
\hline
Probably what you think of & You don't care about this & Philosophy of
representation\\
C, R & & Python\\
Sufficient for small projects & &  \\
Very little hierarchy or structure & list comprehensions, etc. &
tightly couples data w/methods \\
% no coupling in procedural  
\end{tabular}
\end{center}
\end{table}

Consider an object \texttt{Book}. In \texttt{R}, this would have a bunch of
characteristics like ``author'' and ``pages''. In OOP, every book
that exists is an instance of the class ``books.'' In fact,
\emph{everything is an object} in Python: numbers, strings,
classes. Think of classes as a way to give your object some
structure. 

% subtype polymorphism means that the humble ``+'' can do totally
% different things to strings vs. integers. Some (like
% \texttt{__lt__}) can even be defined

\subsection{Why OOP?}
To stay DRY: Don't Repeat Yourself. Ultimately, you will be writing
less code using OOP--even though it won't seem that way at first. A
lot of it comes down to being able to say, ``This is a
\texttt{thing}. It should have \texttt{this} info and \texttt{these}
methods.'' Any code can be written procedurally \emph{or} as OOP, but
the choice is which is simpler. 

\subsection{Inheritance} % this may be what sells OOP this best
% see code from today that begins with ``class Animal(object:''
% the filename is polymorphism.py

Python is a method-passing language, which means it does something
called ``dynamic dispatch.'' Say we defined:

\begin{verbatim} 
class Animal(object): 
  def __init__(self,name): 
    self.name = name

class Cat(Animal): 
  def talk(self): 
    return ``Meow!''
\end{verbatim}


So when if we set \texttt{a} equal to \texttt{Animal(``Fido'')} and
\texttt{b} equal to \texttt{Cat(``Sally'')}, then when we call
\texttt{b.talk()}, it will search within class Cat first, and then
class Animal if it doesn't find anything. % Note that this is ex45 in
                                % Shaw's book.

This is known as \emph{polymorphism}, but it won't mean much to you
until you practice.

As the example \texttt{sports.py} will show, there is a trade-off
between generality and specificity in how much you want a subclass to
inherit from a superior class. Practicing this will force you to think
about:
\begin{itemize}
\item What should exist at the object level?
\item How should those things interact with each other?
\end{itemize}

\textbf{Pro-tip:} Move as far away from \texttt{global} variables as
possible.

\newpage
\section{Pragmatic Programming}

\subsection{Data types}
There are three data types in Python, shown in Table \ref{datatypes}.

\begin{table}[h!]
\begin{center}
\caption{Data Types}
\label{datatypes}
\begin{tabular}{|c|c|c|}
\multicolumn{1}{c}{Name} & \multicolumn{1}{c}{Code} & \multicolumn{1}{c}{Details} \\
\hline
tuple & \texttt{()}, \texttt{a=(1,2)} & immutable, i.e. you can't change it \\
\hline 
list &  \texttt{[]}, \texttt{a=[], a[0] = ``foo'', a[1]=``bar''} & \\
\hline
dictionary & \texttt{\{\}}, \texttt{a[``foo'']=''bar''} & use of keys
\\
\hline
\end{tabular}
\end{center}
\end{table}

% code review

\subsection{Exceptions}

Any time Python ``explodes'' it is because an exception has been
raised. \texttt{Exception} is a class, and all types of exceptions
inherit from this class. Any exception based on the class
\texttt{Exception} will have the attributes \texttt{msg} (what prints
when the exception is raised) and
\texttt{stack trace} (which identifies where the error occurred). Note that ``raise'' is a technical term in
Python. 
% see code inclass3.py

\begin{verbatim}
class CustomException(Exception): # inherits from Exception
  def __init__(self, value):
    self.value = value
  
  def __str__(self):
    return self.value

def i_call_a_function_with_errors():
  try:
    print "Calling a function...."
    #function_with_generic_error()
    #function_with_custom_error()
    #function_with_unknown_error(1)
    print "Tada!"
  except CustomException as inst: # `as' gives us access to the exception
    print "Custom Error Caught! Error({0})".format(inst.value)
  except: # any exception is caught, even ones you don't know about
    print "Default Error Caught!"
  else: # if nothing broke, then run this block
    print "No error raised."
    traceback.print_exc() # this prints the traceback
  finally: # this block is always run
    print "Goodbye!"
  
def function_with_generic_error():
  raise Exception, "Foo!" # this method doesn't know what to do with
the exception
  
def function_with_custom_error():
  raise CustomException, "Foo Bar!" # this will be handled in the
function above}
  
def function_with_unknown_error(foo):
  foo.bar()
\end{verbatim}

Caught exceptions are ones that keep the user from doing something the
programmer didn't want them to do. Uncaught exceptions cause
problems. 

In a Python test suite, you set up a \texttt{try} block to run some
code where you think a problem might happen. As soon as an exception
is thrown in your \texttt{try} block, it doesn't try anything else. 

\textbf{Rule of thumb:} If you know what to do with an error, handle
it. If you don't, let it percolate up. 

Exceptions are something that you've been dealing with up to now but
(maybe) didn't even know about. 

\subsection{Algorithms}
Having an instict for better and worse algorithms will let you know
whether solving your problem will take hours or years.\footnotemark
\footnotetext{``No one in stats gets a Ph.D. without coding and the only
  way people get Ph.D.'s without coding in our discipline is because
  there are some dumb people who have made their way up, but those
  guys are going to die soon.'' -- Scott DeMarchi.} An algorithm can
be defined as ``a series of steps that achieve a desired outcome.''

Consider the task of sorting a list. This is not a problem you will
ever have to solve because better people have already done it, but
it's easy to visualize. We use algorithms to solve hard problems. In
mathematics it's known as an NP problem. (This is just nerd's way of
saying ``hard.'') Most problems in game theory are NP complete. 

In political science, you might be using a data set with 18 million
events. But the thing you're interested in might require
$18,000,000^2$ or even $18,000,000^3$ observations. 18 million cubed
is a big number.

To put this in a Python context, say we want to sort a list $[y_i,
... y_n]$. A terribly inefficient way to do this would be to randomly
shuffle and then check to see if they're in order. There are $n!$ ways
to shuffle the list, so the problem grows factorially. 

In programming, we use ``big O'' notation to talk about
complexity. The $O$ means $\forall x f(x) < cf(x)$. So if we have a
problem $n^2 + n$, we denote it $O(n^2)$ and call it ``quadratic
complexity.'' Complexity of the class $O(n)$ is called ``linear,'' and
so on. It tells us how the number of operations goes up as $n$
grows. We use it to speak about \emph{average} complexity of
problems--after all, we can only speak in averages. 

% look these up because there are great visualizaitons


\subsubsection{Selection sort}
\begin{verbatim}

k = 0, L= [] \\
Loop through n-k \\
    find smallest number at j \\
    swap L[k] with L[j] \\
    k++

              [5,1,15,7,111]  # of things to check
Rd 1: k=0,j=1 [1,5,15,7,111]    5
Rd 2: k=1,j=1 [1,5,15,7,111]    4
Rd 3: k=2,j=1 [1,5,7,15,111]    3...
Rd 4: k=3,j=3
\end{verbatim}

\subsubsection{A Brief Primer/Refresher in Discrete Math}
$\sum_{i=1}^{n} i = {n(n+1) \over 2}$, which is a $O(n^2)$ (quadratic)
complex problem. How do you prove it? 
\begin{eqnarray*}
\sum_{i=1}^{n} i &=& n + (n-1) + (n-2) + ... + 3 + 2 + 1 \\
2 \times \sum_{i=1}^{n} i&=& n+ (n-1) + ... + 2 + 1 + 1 + 2 + ... + (n-1) + (n - 2) \\
&=& (n+1) + (n+1) + ... \\
&=& n(n+1) \\
\end{eqnarray*}

Go back to the example above. How many tries would the best case take?
5. The worst case? 5. And the average case? You guessed it--5. 

\subsubsection{Merge Sort}

This method would take the list to be sort it, split it in half again
and again until they were all disaggregated to the unit level. It would sort those and
reassemble (merge) them into a two-ple (get it?). This reduces the
list sort to an $O(n log(n))$ problem. (In this case we're not even
throwing away little numbers or constants from the $n$.)

Try this out. Come up with a list of numbers, maintain a spot in
them. It's fairly easy to sort two things and interweave them. It will
take $n log_2(n)$ sorts. 

% compare this to mapreduce() in Python, which does log(n) aggregation
% steps in parrallel


\subsubsection{What should I care about in optimizing an algorithm?}
Some people will care about the time it takes. Others will care about
the (memory) space it take. Most people who care about memory work on rocket
ships or microwaves. Merge sort takes more space than selection sort,
but uses less time. As with anything in life, it's a tradeoff. 

\subsubsection{Back to NP}
P means that the problem is in polynomial time class--not that your
algorithm is in P time (it may be exponential) but that the ideal
answer is in P time. Computational game theorists sometimes look at
\emph{whether} something is solvable in polynomial time. If it isn't,
we call it $NP$ time--that is, not solvable. If you can prove that
$P=NP$, you can break all cryptography in the world, make a lot of
money, and retire. If somebody says a problem is ``NP,'' that means
don't waste your time. (Actually they'll say ``NP-hard,''
``NP-complete'' and so on--that just means someone way more
mathematically inclined than us has proven it so; don't argue.)

\subsubsection{Quick Sort}
There is a fourth sort, which we will not get into here, that most
programs actually use. Quick sort is, on average $O (n log n)$, but
its worst case is $n^2$. Again, it all comes back to what you care
about this.

% look it up on Wikipedia. Nothing we're doing is so esoteric that
% someone hasn't covered it better than we could
% http://www.sorting-algorithms.com/

\begin{table}
\begin{center}
\caption{How long will it take to find $y_i$?}
\begin{tabular}{ccccc}
Method & List & Worst & Best & Average \\
\hline
Naive & $[y_1, ..., y_n]$ & $O(n) $ & $O(1)$ & $O({n\over2})$ \\
Binary & sorted & O(log n) & \\
\end{tabular}
\end{center}
\end{table}

Don't stress too much about actually computing complexity
classes--they've already been computed for almost anything interesting
enough that you'd want to work on it. Just know how to choose between
them when presented with options. 

Remember that there is a trade-off between how much time you spend
programming the algorithm and how much time it will save you. In
general, think about how many times you will plausibly be running the
algorithm. Most working programmers look for ``satisficing''
solutions--fast enough and no faster. 
 
% ~\\
% Homework:
% \begin{enumerate}
% \item generate arrays of numbers
% \item sort them
% \item count the time
% \item graph it (N on X axis, time on Y)
% \end{enumerate}

% Implement the sort algorithms first. Everything else will be
% trivial.


\section{Data Structures}
\subsection{A Note on Naming}
In Python, use UpperCamelCasing to name classes and lowerCamelCasing
to name variables or methods. Variables that are member variables of a
class are preceded by two underscores: ``\_\_oneTwo''. These are
conventions that are meaningful to Pythonistas, and should be
meaningful to you as a reader of code and a contributor to projects.

\subsection{Review of Sorting Algorithms}
Take home points:
\begin{itemize}
\item Use \texttt{time.clock()} to benchmark runtimes.
\item Know what your data looks like. (Are there many unique values?)
\item Bubble sort is bad.
\end{itemize}

\subsection{Intro to Data Structures}
Why would you use one data structure over another? Because of what you
want to do with it. There are several classes of data structures, with
many variations on each class--look at Wikipedia. Let's talk about how
lists and arrays work, and when you should use them. We typically care
about how quickly we can add, delete, or find something. (See Table
\ref{data_structures}.)

\begin{table}
\begin{center}
\caption{Data Structures and Their Usefulness}
\label{data_structures}
\begin{tabular}{cccccc}
& Arrays & Lists & Queues/Stacks & Dictionary & Trees  \\
\hline
Add & O(1) or O(n) & O(1), always & O(1)  & O(1) & O(log n)\footnotemark \\ 
Delete & O(n) & O(1) &  O(1) & O(1) & O(log n) \\
Find & O(n) & O(n) & O(n), but we don't care & O(n) & O(log n)  \\
\hline
\footnotetext{O(log n) time for a balanced binary search tree. The
  same goes for deleting.}
\end{tabular}
\end{center}
\end{table}


\subsection{Arrays}
Before getting into the use of specific data structures, it is
important to think about how they are represented on your
computer. Your memory, sometimes referred to as RAM, is known as the
machine's ``heap'' in computer science. When we tell the computer that
an object has a certain schema, it allocates a certain amount of space
based on how much space an object of that schema may take. 

An array looks like \texttt{a = [1,2,''a'']}. When you do this, your
computer gives the object \texttt{a} exactly 3 spaces (an imprecise
term for the moment) of memory. We can index these things quickly
because we are keeping track. However, the tradeoff is that adding
something to list list is difficult. If we run
\texttt{a.append(``b'')}, the computer creates a whole new object with
four spaces and deletes the old one. The actual process of renewing
the list over and over can be very time consuming. 

\subsection{Lists (or ``Vectors'')}
Lists are arrays that do not allow direct indexing. Rather than
allocating a block of memory, a list is stored as items that each
point to the next item in the list. The only way to get the second
thing out of memory is to get the first thing out of memory and ask it
where the second thing is hiding. If we want the fourth (or the
$n^{th}$ element, that can become a lot of asking. To find something,
we have to use an algorithm, typically starting at the middle and
working our way out. 

The benefit of a list is that you can add things to it
indefinitely. Furthermore, they can grow indefinitely as well because
Python does not allocate a fixed space for the list when it was
created. 

\subsection{Queues}
(To build a queue, you need a list or an array. We're working our way
up.)

Queues only support enqueue or dequeue (adding something to the end,
or removing from the front, of a line). You could use either an array
or a list to store these. Which is better? A list, because we aren't
going to be doing searching, just adding and deleting. 

Queues are managed by the rule of FIFO (First-In-First-Out).

This is not a natively supported data type in Python, so if you want
one you'll have to build it yourself, or \texttt{import from
  collections deque}. Python also does not differentiate between arrays and
lists, but in \texttt{R} it does make a difference. 

\subsection{Stack}
As the name would suggest, this can be best visualized as a stack of
things. Stacks support LIFO (Last-In-First-Out). Again, you only add
or remove things. Things can only get pushed onto or popped off of the stack. 

\begin{table}
\begin{center}
\caption{A Stack}
\label{stackfig}
\begin{tabular}{|c|}
\multicolumn{1}{c}{} \\
\hline
``b'' \\
\hline
``a'' \\
\hline
2 \\
\hline
1 \\
\hline
\end{tabular}
\end{center}
\end{table}

As with lists and arrays, Python conflates the difference between
stacks and queues. If you're working with data of known size, this
isn't a huge problem. 

What's going on under the hood here? When you call a function, it gets
put in your stack. Your computer has a huge chunk of memory for the
stack. In the pseudocode below, the user calls \texttt{a}, which in
turn calls \texttt{b}, which in turn calls \texttt{c} and returns
1. How does something get popped off? A function is popped off when it
returns. A recursive function just puts itself on the stack over and
over again.\footnotemark \footnotetext{``When you try to push something onto the stack when the
stack is full, you have overflown the stack.''} Fortunately, Python tries to help you (in the terminal at
least) and tries to keep you from overflowing the stack with recursive
functions. 

\begin{verbatim}
def a():
    b()

def b():
    c()

def c():
    a = 1

>>> a()
>>> 1
\end{verbatim}

When a problem happens, the stack trace is returned, which will tell
you all of the functions that failed to catch it. (Tattle-tale.)

Because a computer core only does one thing at a time, a stack is an
efficient way to handle it. An eight-core computer has eight
stacks. 

\subsection{Dictionary/Hashtable}

Dictionaries provide a quick lookup for arbitrary things. It is called
a hashtable because it uses a hash function: it assigns a unique
identifier (typically a number) to the object that the arbitrary key
should point to. 
\begin{verbatim}
a = {}
a["foo"] = "bar"
\end{verbatim}

When we do the above code, we tell the computer ``a + `foo' = memory
location.'' The pointer that gets us from input to output is the hash
function. The main point here is that you shouldn't be doing a lot of
searching through a dictionary, because it is bad for that. What
dictionaries are good for is adding and removing a lot of things very
quickly, and to access them (but not sort them). This is one of the
most commonly used data structures for a reason: you don't know how
many data points you will have, but you want to get to them
quickly. Understanding lists and dictionaries are the most important
points for today. 

\subsection{Trees}

Trees come in many shapes, but it is easiest to visualize a binary
tree (think game theory; each node has at most two branches). The top is the root, anything that doesn't
have another node after it is a leaf, and the distance between the
root and the farthest leaf is the height of the tree. Depending on the
properties you enforce for your tree, you can get interesting
patterns. If you put numbers into your tree in a certain way, you can
access them very quickly: everything to the right is bigger than its
parent node, everything to the left is smaller than its parent node. 

The nodes of the tree are stored with pointers to the other
items. Each node points to its right and left child (which may be
empty if the node is a leaf). 

You can think of a list as a degenerate case of a tree. These are
pretty easy to create, so many people make their own. Useful terms to
keep in mind are ``max heap'' and ``min heap,'' in which the greatest
number (or the smallest number) is on top. There is also an importance
heap, which is beyond the scope of the current discussion.

\subsection{Graphs}

Graphs are a computer scientist's favorite thing. They consist of
nodes and edges. Nodes can contain or not contain things, depending on
what you want the graph to look like. Edges can be weighted or
unweighted, depending on your problem. Weighted edges are interesting
(an NP hard problem) in the travelling salesman problem. In this
problem, the edges are weighted by the mileage between the
cities. Edges can be directed or non-directed, which governs how the
graph can be traversed. A graph is ``cyclic'' if you can get back to
your starting point for any smaller portion of the graph.

To make things more confusing, people like to abbreviate. A DAG is a
Directed Acyclic Graph. Trees can be thought of as degenerate cases of
graphs, and are by definition DAGs.

No one uses graphs in Python, but you will for
your homework. You will handle them as objects. Another way is with an adjacency matrix (which can be
weighted or binary). 


\section{Input/Output}

% begin with code review, as usual

% things to talk about
% hasCycle
% O(n) remove

% there is a way to do reverse in O(n) time with one extra memory
% space

% exec() keyword in python is comparable to source() in R
% BUT "if you are using the exec command, something has almost
% definitely gone wrong"

After a code review of last week's homework, we are done talking about
algorithms and complexity for their own sake. Just know that as your
problem gets big, you will feel the pain of picking a bad algorithm. 

The slowest part of a computer is a hard drive, so most algorithms are
optimized not only by time, but by minimizing the work you're doing on
the hard disk. Given that everything else in your computer happens at
the speed of light, going around a hard drive at 5000 or 7200 rpm is a
snail's pace. This is why ``defragging'' would speed up a machine, and
why the hard drive is the first thing to go bad. Wider availability of
solid-state drives (SSD) will change the things we have to say, but
for now they are not widespread enough that we can ignore hard disk
read/write time. 

% We won't go around proving the things we say today are true. People
% have written other books on it. Cite those.

Reading from a disk is really no different than reading data from the
internet, or calling \texttt{print}. The machine is calling something
called \texttt{stdout} and putting it on your screen. (Recall that in
Zed Shaw's Python book you used \texttt{stdin} by calling
\texttt{raw\_input()}.)

\subsection{filestuff.py}
There is a graceful way to open files. Use it. 

(Note that the readfile.txt called by filestuff.py is just the text
from our department home page.) 

When you see ``\texttt{for l in f:}'' in Python, Python will
understand that to mean using the native linebreaks in the file. 

\texttt{f.seek(0)} is how we move the pointer back to the $0^{th}$
byte of the file. A byte is one character in space representation. 

\subsection{urlgrabbing.py}
URL: \emph{U}niversal \emph{R}esource \emph{L}ocator/Lookup

HTML is a contract for how you should structure the document so the
web browser knows what to draw on the screen. When we talk today,
we're discussing ``html in its proper form.'' This is an ideal that
does not exist on 99 percent of the web. Fortunately, other people
have solved this problem for you--use libraries.

Here's an example of XML and what it allows you to to:
\begin{verbatim}
<thing>
  <nested_thing>

  <\nested_thing>
<\thing>
\end{verbatim}
Notice that the identifiers inside the ``<>'' are arbitrary. You can
find actual HTML anywhere by right-clicking and selecting ``view page
source.''

Here's a stylized example:
\begin{verbatim}
<html>
  <head>         # not visible
    <title>
  <\head>

  <body>
    <h1>Foo</h1> # a header block (headers go from h1 to h6,
decreasing in size)
    <p>bar<\b>   # a paragraph tag
    <a href="google.com">google</a> # a hyperlink
    <img src = "pic.png"> # an image
    <div>        # a divider
    <table>      # care to guess?
    <span>
    <hr>

  </body>
</html>
\end{verbatim}

Libraries for parsing HTML are available in almost every language. A
good one in Python is BeautifulSoup.\footnotemark \footnotetext{In OS
  X, install from terminal using \texttt{easy\_install BeautifulSoup}
  or \texttt{pip install BeautifulSoup}. }

Before you go running off to spider everything, a few things to note:
\begin{itemize}
\item sitemap.xml = what to index
\item robots.txt = what not to index
\item rel=``nofollow'' = if you're a crawler, don't go there
\end{itemize}

Your laptop on an automated search is much faster than the way people
normally use the internet. Use a delay--a second or two will suffice. 


\section{More Applications: Twitter API}
% code review as usual
% Discuss class searching in HTML/CSS
% <a id= "foo" class = "bar bas">
% a is the tag
% id is unique
% class is not unique
% a = all links
% #foo = id == foo
% .bar = class == bar
% a.bar = tags with class bar
% a.bar.bas

\subsection{Things we didn't discuss last week (on purpose)}

HTML has a bunch of encoded versions of characters, such as
\texttt{\&nbsp;}. Every program that interacts with the web will have a
library for dealing with HTML entities. Anything that starts with
``\&'' and ends with ``;'' is an HTML entity. 

Another problem that you may run into is \emph{relative} URLs. You do
not actually have to specify the root of a page to access it; HTML
just assumes you want the same root that you are on currently. The
Python library \texttt{urlparse} can deal with this through the
command \texttt{urljoin}. 

\subsection{Application Programming Interfaces (API)}

You already know what these are, even if you do not realize it. An API
can be thought of as a contract for how a message is passed. We
thought about this on a smaller scale when we implemented
functions. For example, in Python the code
\begin{verbatim}
def foo(bar):
    print bar
\end{verbatim}
has a contract to accept \texttt{bar} and return the printed form. 

There are different types of API's. One is SOAP and it is bad. If you
have the choice, do not use it. The preferred option is REST, which
stands for ``Representational State Transfer.''\footnotemark
\footnotetext{As an interesting piece of history, the same individual
  conceptualized REST and HTTP. Both are contracts for information
  delivery.} 

\subsubsection{REST}

In a REST API, all things are objects. We do not mean this in quite
the same sense as OOP, but it means that we are doing things to other
things. If this makes perfect sense to you, do not worry about the
alternative. All of the things that the API works with (users, pages,
coordinates, etc.) have unique
URI's, which saves a lot of confusion. All things also have a
representation; you know about the HTML representation, but you could
also get the XML or the JSON representation. The important point is
that we can elicit different representations of the same object from
the API. 

Note that the Twitter API has a rate limit of 350 requests per hour. 

\subsubsection{HTTP}

HTTP is a way of representing object. Important types of objects
include clients (e.g. browsers, urllib2, etc.) and servers (the thing
you crashed if you failed to put a \texttt{time.sleep()} command in
your last homework). Observe that \texttt{git://} has a different
representation that \texttt{http://}, which is itself different from
\texttt{https://}. 

\subsubsection{CRUD}
You will encounter this acronym when discussing what an object can
do. It stands for ``Create, Read, Update, Delete.'' When you access
this object through HTML, we are using its \emph{read} attribute. HTTP
has four actions: \emph{G}et, \emph{P}ost, \emph{P}ut,
\emph{D}elete. Most everything you have done on the web up to this
point is post (which corresponds to create) and get (which corresponds
to read).

On Twitter, when you read a tweet, you are getting it. When you post
it you are creating it. When you delete
it, you are deleting it (makes sense, right?). All of these things
happen to the same URI, and someone has packaged them into a nice neat
library for you. 

A URI is made up of 5 things: domain, protocol, path, query string,
hash. You should already be familiar with the domain, protocol, and path. Query strings start with ``?'' and are concatenated with
``\&''. Hashes start with--gasp--``\#''. 

There's a lot going on on the server side too. The server will depend
on you giving it requests that have a representation (e.g. MIME,
XML). It will give you back the info you ask for, along with a code
(see Table \ref{server_codes}).

\begin{table}
\begin{center}
\caption{Server Codes}
\label{server_codes}
\begin{tabular}{lcc}
Code & What it means & What probably happened \\
\hline
200 & OK & Something good \\
500 & Server error & \\
503 & Temporarily down & \\
404 & Not found & Page deleted \\
420 & Enhance your calm & exceeded ratelimit \\
418 & I'm a little teapot & April Fool's \\
\hline
\end{tabular}
\end{center}
\end{table}

\subsection{The Twitter API}

First, you have to register an application with Twitter via
\texttt{http://dev.twitter.com/}. You can register before your code is done. They will give you the
keys you need, which will keep people from using your app, enables the
permissioning system, and keeps track of your rate limiting. 

You will need to install the library \texttt{tweepy}. 

You will also have been given a Consumer Key and a Consumer
Secret. Tweepy will need those. 

Create an API object, which will keep you from having to do all of the
things we talked about above as far as getting, posting, etc. 

See \url{https://dev.twitter.com/docs/api} for all of the cool things
you can do with your API.

% sopa_tweets = api.search('#sopa')
% sopa_tweets[0]

This has been a brief introduction, because you will learn it best by
doing it. 

\section{Databases}
This is a huge topic, but fortunately you can get functional very
quickly and learn as you go. Most of the actual work that goes on with
databases is plain vanilla, but there are some strange extreme-use
cases that take some know-how and experience. 

There are a number of different kinds of databases:
\begin{itemize}
\item Relational database (what we'll do): Mysql, sqlite, SQL Server,
  PostGRES (many of these are free) 
\item Key-value Store (the other main one): MongoDB, TokyoCabinet,
  Redis, Cassandra, CouchDB
\item Variations on these two 
\end{itemize}

We use databases instead of, e.g., CSV files because CSV's are not
optimized for fast searching or any other such thing that we
actually care about. Smart people have spent a lot of time optimizing databases
for the kind of searching we're doing since the 1970's. 

When we talk about a database, two concepts are important:
\emph{schemas} and \emph{rows} of data. The schemas in a database are
strict. By this, we mean that the database will only return valid data
(i.e. no strings when we want a number--if we set it up that way). The
simplest way to think about databases, at least at first, is like a
spreadsheet: a table with rows and columns. The schema defines what
columns you have, how they are named, and what \emph{type} they
have. In addition to type (e.g. \texttt{int}, \texttt{str}--actually
SQL uses \texttt{varchar} instead of string), you can
also define the allowable length, whether or not \texttt{Null} is
allowed, and whether or not uniqueness is required. You may not want to allow nulls, for example, when building a
Twitter database, so that you know every row you look at has a valid
username. Uniqueness prevents (or allows) duplicate information. Often
joint uniqueness is helpful when dealing with large data, for example
name and social security number as opposed to name alone. Note that
SQL databases \emph{require} you to have a unique (or joint unique)
identifier for each row. 

Why are they called `relational databases?' Good question. Mostly
because people who work on databases are striving to ``normalize'' the
data. Say we have a database with three columns, and a value in one of
them changes. The concept of normalization means that the nominal
identifier in our database (say, a name) is also assigned a unique
identifier (usually a number), so that if a change is made to the
nominal identifier, we can still keep track of what happened. It also
allows us to make changes quickly, and is optimized for both time and
memory.

\subsection{Schemas}
Relational databases assume you're going to normalize, so they make it
very quick to organize tables. Consider the example in Tables
\ref{aDatabase} and \ref{aNormalizer}.

\begin{table}[htbp]\centering
 \caption{An Example Database}
\label{aDatabase}
\begin{tabular}{l|c}\hline
Book\_Title & author\_id \\
\hline
Bible & 1 \\
Farenheit 451 & 2\\
October Country & 2\\
Cat's Cradle & 3 \\
Catch 22 & NULL \\
\end{tabular} \\
\end{table}

\begin{table}[htbp]\centering
 \caption{A Normalizing Table}
\label{aNormalizer}
\begin{tabular}{l|c}\hline
1 & God \\
2 & Bradbury \\
3 & Vonnegut \\
4 & Tolkien \\
\hline
\end{tabular} \\
\end{table}

Now say we join them using an \emph{inner join}. That would return
Table \ref{innerjoin}:
\begin{table}[h!]\centering
 \caption{An Inner Join}
\label{innerjoin}
\begin{tabular}{l|c}\hline
Bible & God \\
F451 & Bradbury \\
OC & Bradbury \\
CC & Vonnegut \\
\hline
\end{tabular} \\
\end{table}
There are also left and right \emph{outer joins} that work similarly,
but will return NULL for objects that it cannot match.

\subsection{ORM}

ORM stands for \textbf{o}bject-\textbf{r}elational \textbf{m}odel. It is a mapping of objects to
database tables. We will look at what the SQL commands do, and how we could right them
on our own. If you have questions about SQL, the internet is a vast
and wonderful place to find answers. 


\section{Intro to Classifiers}

We will discuss the most mathematically tractable, easiest to
understand classifier: naive Bayes classifiers. They are also the
easiest to implement, and happen to work shockingly well on lots of
classifiers. The output of a classifier is the probability that a
thing belongs in a set. A simple introductory example is the
classification of email as spam/not spam. 

To build a classifier, you first need to understand the following
things about your problem domain: 
\begin{itemize}
\item what you want to classify (the \emph{document})
\item \emph{features} of the document
\item which \emph{features} you want to emphasize for classification
  (i.e. which will be good predictors)
\item the \emph{categories} you wish to sort to
\end{itemize}

\subsection{Bayes' Rule: Review and Example}
Bayes' Rule is about taking some conditional probability and then
using it to calculate other conditional probabilities.

\begin{eqnarray*}
P(\text{spam}|\text{features}) &=&
{{P(\text{features}|\text{spam})P(\text{spam})} \over P(\text{features})}
\end{eqnarray*}

We will use what is called the ``bag of words'' approach. In the
example above, our features are word counts. The ``bag of words''
approach only deals with individual words rather than phrases. We
don't want to evaluate all of the $n$-grams in a 1,000-word text
document because the combinatorics are huge. To deal with this, we
look at the text with the assumption that words appear
independently. Note that this is a huge assumption that does not hold
in reality, but it makes the math clean and works fine for most
cases.\footnotemark \footnotetext{Recall our earlier discussion about
  ``most cases.''} 

\begin{eqnarray*}
P(\text{spam}|\text{"male enhancement"}) &=&
{{P(\text{"male enhancement"}|\text{spam})P(\text{spam})} \over
  P(\text{"male enhancement"})} \\
&=& {{P(\text{"male"}|\text{spam})P(\text{spam})P(\text{"enhancement"}|\text{spam})} \over
  P(\text{"male"}) P(\text{"enhancement"})} 
\end{eqnarray*}

Here is an example from the Stanford AI course:
\begin{table}[h!]
\caption{Spam vs. Ham Phrases}
\label{spam}
\centering
\begin{tabular}{c|c}
Spam & Ham \\
\hline
Offer is secret & Play sports today \\
Click secret link & Went play sports \\
Secret sports link & Secret sports event \\
& Sports is tody \\
& Sports cost money 
\end{tabular}
\end{table}

Let's use the data in Table \ref{spam} to estimate the conditional probabilities:
\begin{eqnarray*}
P(\text{"secret"|spam}) &=& \frac{3}{9} \\
P(\text{"secret"|ham}) &=& \frac{1}{15} \\
P(\text{spam|"sports}") &=& {
  P(\text{"sports"|spam}) P(\text{spam}) \over
  {P(\text{"sports"}|\text{spam})P(\text{spam})P(\text{"sports"}|\text{ham})
    P(\text{ham})}} \\
&=& {\frac{1}{9} \times \frac{3}{8}} \over {\frac{1}{9} \times
  \frac{3}{8} + \frac{1}{3} \times
  \frac{5}{8}}
\end{eqnarray*}

Notice that this is probability for an individual word. Also, realize
that our preferences over Type I versus Type II errors might cause us
not to make 50 percent our threshold for categorization. 


% insert a more complicated example here if desired

\subsection{LaPlace Smoothing}
One problem with Bayes' Rule is that in the naivest of cases it does
not know how to handle something it has not seen before. A LaPlace
smoother adds a small constant and re-adjusts according to the rules
of probability to ensure that there are not zero probabilities that a
message is spam. (This is akin to adding .001 if you are trying to log
event counts, etc.)

Compare the naive and Laplace-smoothed estimators below:
\begin{eqnarray*}
ML p(x) &=& {count(x) \over N} \\
LS p(x) &=& {count(x) + k \over N + k|x|}
\end{eqnarray*}
Notice that the LS estimate converges to the ML estimator when the
number of categories $|x|$ is large. $k$ is arbitrary--choose what
works. 1 is usually a good place to start.

% redo complicated example above with LaPlacean Smoothing

\subsection{Pseudo-Implementation}

A minimum-working example of a classifier consists of three things: a
hash of words, a training method, and a classifier method.

Some other terms to keep in mind when implementing are:
\begin{itemize}
\item word breaking: a method to recognize that Doctor, Dr., and Dotcor all
represent essentially the same thing to the human mind
\item word stemming: a method to recognize that ``angry'' and
  ``anger'' each consist of ``angr-'' and are thus functionally
  equivalent for purposes of classification (often this results in a
  substantial improvement in probability calculation)
\item stop words: words like ``in, at'' and ``the'' can lead to
  spurious results and take up space in your data storage, so most
  folks ignore them
\end{itemize}

There are libraries for all of these made by people with a lot of
experience--use those. (Note that all of our examples today are
English-specific; word stemming may not work in many languages, and
word breaking can be extremely hard.) 


\section{Linear Programming}

Linear programming is an advanced topic from a CS perspective, because
the way that it works is technically difficult. However, there are
other places to look if you are interested in learning that aspect. We
will discuss the interesting part from a data modelling perspective. 

Think of linear programming as an optimization problem. That is, we
seek a solution to \emph{minimize}, \emph{maximize}, or
\emph{equality}, subject to some constraints. If you have solved a set
of equations for a variable, you have done this. If you have ever
worked on game theory or economics, you have dealt with maximization
(and possibly minimization). In calculus you can also sometimes format
problems as linear problems. For computers, these problems are easily
solvable. We will work with \texttt{CPLEX}, or Gnu Linear Programming
Kit. 

\subsection{Example: Paintings}
Consider two types of paitings in Table \ref{paintings}. Our
constraint is that we have 16 blues, 8 greens, and 5 reds. One way to
solve this would be to draw two-dimensional indifference curves. 

\begin{table}[h!]\centering
 \caption{Two Types of Paintings}
\label{paintings}
\begin{tabular}{cc}\hline
Price & Supplies Required \\
\hline
\$30 & 4 blue, 1 green, 1 red \\
\$20 & 2 blue, 2 green, 1 red \\
\hline
\end{tabular} \\
\end{table}

We can also set up our constraints as a series of equations:
\begin{eqnarray*}
\text{maximize:  } 3x + 2y \\
\text{blue:  } 4x + 2y  &\leq& 16 \\
\text{red:  } x + y  &\leq& 5 \\
\text{green:  } x + 2y  &\leq& 8 
\end{eqnarray*}

As you can see by setting this up in \texttt{CPLEX}, once you can set
up a problem as a set of linear equations, it is trivially easy to
solve. The cleverness is all in setting up the problem.\footnotemark
\footnotetext{The same could be said of practically all normal-form,
  finite game theory problems.}

% Pulp is a Python wrapper for mathprog

\subsection{Airlines}
Now consider another example, a fictional airline industry. We are
trying to determine the minimum number of airlines to deploy from the
set in Table \ref{airlines} to get full global coverage. \footnotemark
\footnotetext{Read this carefully to see what kind of problem we are
  trying to solve and the constraint(s) we are under.}

\begin{table}[h!]\centering
 \caption{Airline Coverage}
\label{airlines}
\begin{tabular}{c|c}\hline
Airline & Continents Covered \\
\hline
A & NA, SA, Af \\
B & E, Af, As \\
C & NA, Au, E \\
D & SA, As, E \\
E & Af, As, Au \\
F & NA, E \\
\hline
\end{tabular} \\
\end{table}

\begin{eqnarray} % TODO: check this to see if it's actually right
\label{airlineeqn}
\text{minimize:  } A+B+C+D+E+F \\
\text{NA:  } A + C + F  &\geq& 1 \\
\text{SA:  } A + D  &\geq& 1 \\
\text{Af:  } A + B + E  &\geq& 1 \\ 
\text{E:  } B + C + D + F &\geq& 1 \\ 
\text{As:  } B + D + E  &\geq& 1 \\
\text{Au:  } C + E  &\geq& 1 
\end{eqnarray}

Notice that this set of equations fully specifies the
problem we are trying to solve. We can write this as a linear
program and see that ABC and DEF are both solutions (there may be
others).

\subsection{Bipartite Matching}
This example is intended to illustrate that networking can also be
analyzed as problem for linear programming. (See illustration in
Josh's notes.) % TODO: add illustration

\subsection{Feasibility}

In all of the examples above, we sought integer solutions. Sometimes
finding the integer solution can take a great deal of time (or even be
infeasible), but you can still find the frontiers of the feasibility
space if you search for a non-integer solution. Just something to keep
in mind. 
 

\pagebreak
\onehalfspacing
\bibliographystyle{/Users/mcdickenson/Documents/apsr}
\bibliography{/Users/mcdickenson/Library/textmf/bibtex/bib/RefLib.bib}

\pagebreak
% \section*{Tables}





\end{document} 
